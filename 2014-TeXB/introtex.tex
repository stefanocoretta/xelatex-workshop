\documentclass{beamer}
\usepackage{dtklogos}
\title{Introduction to \XeLaTeX}
\author{Stefano Coretta}

\usetheme{Szeged}
\usecolortheme{beaver}


\usepackage{polyglossia}
	\setmainlanguage{english}
\usepackage{fontspec}
	\defaultfontfeatures{Mapping=tex-text}
	\setmainfont{DejaVu Sans}
	\setsansfont{DejaVu Sans}
	\setmonofont{DejaVu Sans Mono}
	\setromanfont[Renderer=ICU]{Charis SIL}
	\newfontfamily\SC{DejaVu Sans SC}
\usepackage{ctable,multicol,multirow,array,listings}
	\lstset{frame=single, numbers=left,breaklines=true,language=[LaTeX]TeX,breakindent=0pt}
\usepackage{graphicx}
\usepackage{gb4e}
\usepackage{etoolbox}
\renewcommand{\eachwordone}{\sffamily\bfseries}
\pretocmd{\glt}{\sffamily}{}{}
	\let\eachwordtwo=\sffamily


	
\setbeamertemplate{caption}[numbered]
\setbeamertemplate{section in toc}[sections numbered]
\setbeamertemplate{footline}
{%
  \begin{beamercolorbox}[colsep=1.5pt]{upper separation line foot}
  \end{beamercolorbox}
  \hbox{%
    \begin{beamercolorbox}[wd=0.333333\paperwidth, ht=2.5ex, dp=1.125ex, center]{title in head/foot}%
      \usebeamerfont{author in head/foot}\insertshortauthor
    \end{beamercolorbox}%
    \begin{beamercolorbox}[wd=0.333333\paperwidth, ht=2.5ex, dp=1.125ex, center]{title in head/foot}%
      \usebeamerfont{title in head/foot}\insertshorttitle
    \end{beamercolorbox}%
    \begin{beamercolorbox}[wd=0.333333\paperwidth, ht=2.5ex, dp=1.125ex, center]{title in head/foot}%
      \usebeamerfont{title in head/foot}\insertframenumber/\inserttotalframenumber\hspace*{2ex}
    \end{beamercolorbox}}
  \begin{beamercolorbox}[colsep=1.5pt]{lower separation line foot}
  \end{beamercolorbox}
}

%%%%%%%%%%%%%%%%
%%% DOCUMENT %%%
%%%%%%%%%%%%%%%%


\begin{document}

\begin{frame}
	\maketitle
\end{frame}

\begin{frame}
	\frametitle{Contents}
	\tableofcontents
\end{frame}

\section{Introduction}

\begin{frame}
	\frametitle{What is \XeLaTeX{}?}
	
\begin{itemize}
\item it is a mark-up language for typesetting and text writing (and more)
\item it is a variant of the \LaTeX{} format that introduced full Unicode support and \texttt{.ttf}, \texttt{.otf} font handling
\item it is implemented by the \XeTeX{} engine, that is derived from the \TeX{} system
\end{itemize}

\end{frame}

\begin{frame}
	\frametitle{How does it work?}
	
\begin{itemize}
\item put the \XeLaTeX{} code in a \texttt{.tex} with the content of your document
\item the \XeTeX{} engine reads and typeset the code
\item a \texttt{.pdf} file with your document is produced
\end{itemize}

\end{frame}

\begin{frame}
	\frametitle{Get it!}
	
\begin{itemize}
\item download a \TeX{} distribution (\textit{distro})
\begin{itemize}
\item MikTeX
\item TeXlive 
\item MacTeX
\end{itemize}
\item a \textit{distro} contains both the systems and the engines, plus software that ease the typesetting process
\end{itemize}

\end{frame}

\section{Typeset}
\subsection{First steps}
\begin{frame}[fragile]
	\frametitle{Document class}
	
\begin{lstlisting}
\documentclass{article}


\begin{document}

Hello.

\end{document}
\end{lstlisting}

\end{frame}

\begin{frame}[fragile]
	\frametitle{Headings}
	
\begin{lstlisting}
\documentclass[12pt,a4paper]{article}

\begin{document}
\tableofcontents
\section{Introduzione}
Text...

\subsection{Premesse}
TExt...

\end{document}
\end{lstlisting}

\end{frame}

\begin{frame}[fragile]
	\frametitle{Title}
	
\begin{lstlisting}
\documentclass{article}
\title{Introduction to linguistics}
\author{John Smith}
\date{}

\begin{document}
\maketitle
...

\end{lstlisting}

\end{frame}
\subsection{Packages and layout}
\begin{frame}[fragile]
	\frametitle{Packages}
	
\begin{lstlisting}
\documentclass{article}

\usepackage{fontspec}
  \setmainfont{Arial Unicode MS}
\usepackage{polyglossia}
  \setmainlanguage{english}
...
\end{lstlisting}

\end{frame}

\begin{frame}[fragile]
	\frametitle{Text layout}
	
\begin{lstlisting}
For a new paragraph, leave an empty row.

This is the new paragraph. To format your text: \textit{italics}, \textbf{bold face}.

For a footnote.\footnote{This is a footnote.}
\end{lstlisting}

\end{frame}

\begin{frame}[fragile]
	\frametitle{Bulleted and numered lists}
\begin{lstlisting}
\begin{itemize}
\item Gender: masculine, feminine
\item Number
  \begin{enumerate}
  \item singular
  \item plural
  \end{enumerate}
\end{itemize}
\end{lstlisting}
\end{frame}

\begin{frame}
	\frametitle{Bulleted and numered lists}
\begin{itemize}
\item Gender: masculine, feminine
\item Number
  \begin{enumerate}
  \item singular
  \item plural
  \end{enumerate}
\end{itemize}
\end{frame}


\section{Linguistics}
\subsection{Standard elements}
\begin{frame}[fragile]
	\frametitle{Numbered examples}
\begin{lstlisting}
\usepackage{gb4e}
...
\begin{exe}
\ex Nel mezzo del cammin di nostra vita
\ex
  \begin{xlist}
  \ex Mi ritrovai per una selva oscura
  \ex Che la diritta via era smarrita
  \end{xlist}
\end{exe}
\end{lstlisting}
\end{frame}

\begin{frame}
	\frametitle{Numbered examples}
\begin{exe}
\ex Nel mezzo del cammin di nostra vita
\ex
  \begin{xlist}
  \ex Mi ritrovai per una selva oscura
  \ex Che la diritta via era smarrita
  \end{xlist}
\end{exe}
\end{frame}

\begin{frame}[fragile]
	\frametitle{Tables}
	
\begin{lstlisting}
\usepackage{ctable}
...
\ctable[caption=Objects,
label=obj,
pos=t
]{lcc}{}{
\FL
 & Given O & New O \ML
VOP & 85 & 65 \NN
VPO & 100 & 147 \LL
}
\end{lstlisting}

\end{frame}

\begin{frame}
	\frametitle{Tables}

\ctable[caption=Objects,
label=obj,
pos=t
]{lcc}{}{
\FL
 & \SC{Given O} & \SC{New O} \ML
VOP & 85 & 65 \NN
VPO & 100 & 147 \LL
}
\end{frame}

\begin{frame}[fragile]
	\frametitle{Interlinear glosses}
	\small
\begin{lstlisting}
\begin{exe}
\ex \gll Mi sdirrup-a-ia cu tutt-a a lap-a. \\
1s.\SC{refl}  tumble-\SC{thm}-1s.\SC{past} \SC{com} all-\SC{f} \SC{def}.\SC{f} Piaggio.Ape-\SC{f} \\
\glt "I tumbled with the whole Piaggio Ape."
\end{exe}
\end{lstlisting}
\end{frame}

\begin{frame}
	\frametitle{Interlinear glosses}

\begin{exe}
\ex \gll Mi sdirrup-a-ia cu tutt-a a lap-a. \\
1s.\SC{refl}  tumble-\SC{thm}-1s.\SC{past} \SC{com} all-\SC{f} \SC{def}.\SC{f} Piaggio.Ape-\SC{f} \\
\glt ``I tumbled with the whole Piaggio Ape.''
\end{exe}
\end{frame}


\subsection{Special functions}

\begin{frame}[fragile]
	\frametitle{Cross-referencing}
\begin{lstlisting}
\section{Methodology}
\label{metho}
Testo...
\begin{exe}
\ex\label{es} Stay hungry, stay foolish.
\end{exe}

\section{Conclusioni}
As I said in section \ref{metho}... In example (\ref{es})...

\end{lstlisting}
\end{frame}

\begin{frame}[fragile]
	\frametitle{Bibliography}
\begin{lstlisting}
\usepackage{natbib}
\setcitestyle{aysep={},notesep={:}}
...
As suggested by \citet[45--47]{dixon2001intro}... This was proved by several works \citep{dryer13,goldsmith1986auto,givon2002syntax}...
...
\bibliography{mybib}
\bibliographystyle{unified}
\end{lstlisting}


\end{frame}






\end{document}