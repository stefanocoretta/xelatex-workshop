\documentclass{beamer}
\usepackage{dtklogos}
\title{Introduzione a \XeLaTeX}
\author{Stefano Coretta}
\date{3 Dicembre 2013}
\usetheme{Szeged}
\usecolortheme{beaver}


\usepackage{polyglossia}
	\setmainlanguage{italian}
\usepackage{fontspec}
	\setmainfont{DejaVu Sans}
	\setsansfont{DejaVu Sans}
	\setmonofont{DejaVu Sans Mono}
	\defaultfontfeatures{Mapping=tex-text}
\usepackage{ctable,multicol,multirow,array,listings}
	\lstset{frame=single, numbers=left,breaklines=true,language=[LaTeX]TeX,breakindent=0pt}
\usepackage{graphicx}
\usepackage{gb4e}
	\let\eachwordone\bf
\setbeamertemplate{caption}[numbered]
\setbeamertemplate{section in toc}[sections numbered]
\setbeamertemplate{footline}
{%
  \begin{beamercolorbox}[colsep=1.5pt]{upper separation line foot}
  \end{beamercolorbox}
  \hbox{%
    \begin{beamercolorbox}[wd=0.333333\paperwidth, ht=2.5ex, dp=1.125ex, center]{title in head/foot}%
      \usebeamerfont{author in head/foot}\insertshortauthor
    \end{beamercolorbox}%
    \begin{beamercolorbox}[wd=0.333333\paperwidth, ht=2.5ex, dp=1.125ex, center]{title in head/foot}%
      \usebeamerfont{title in head/foot}\insertshorttitle
    \end{beamercolorbox}%
    \begin{beamercolorbox}[wd=0.333333\paperwidth, ht=2.5ex, dp=1.125ex, center]{title in head/foot}%
      \usebeamerfont{title in head/foot}\insertframenumber/\inserttotalframenumber\hspace*{2ex}
    \end{beamercolorbox}}
  \begin{beamercolorbox}[colsep=1.5pt]{lower separation line foot}
  \end{beamercolorbox}
}

%%%%%%%%%%%%%%%%
%%% DOCUMENT %%%
%%%%%%%%%%%%%%%%


\begin{document}

\begin{frame}
	\maketitle
\end{frame}

\begin{frame}
	\frametitle{Sommario}
	\tableofcontents
\end{frame}

\section{Introduzione}

\begin{frame}
	\frametitle{Cos'è \XeLaTeX}
	
\begin{itemize}
\item è un linguaggio di \textit{markup} per la compilazione tipografica e la stesura di testi (e non solo)
\item è una variante del formato \LaTeX{} che introduce il supporto a Unicode e ai font \texttt{.ttf} e \texttt{.otf}
\item è implementato dal motore \XeTeX, a sua volta derivato del sistema \TeX
\end{itemize}

\end{frame}

\begin{frame}
	\frametitle{Come funziona}
	
\begin{itemize}
\item in un file \texttt{.tex} si inserisce il codice con il contenuto del documento
\item questo codice è letto e compilato dal motore (\XeLaTeX)
\item l'output della compilazione è un file \texttt{.pdf}
\end{itemize}

\end{frame}

\begin{frame}
	\frametitle{Come ottenerlo}
	
\begin{itemize}
\item scaricando una distribuzione (\textit{distro}) \TeX
\begin{itemize}
\item MikTeX
\item TeXlive 
\item MacTeX
\end{itemize}
\item una \textit{distro} contiene sia i sistemi che i motori di compilazione, più software che facilitano la compilazione
\end{itemize}

\end{frame}

\section{Compilazione}
\subsection{Primi passi}
\begin{frame}[fragile]
	\frametitle{Classe del documento}
	
\begin{lstlisting}
\documentclass{article}


\begin{document}

Ciao.

\end{document}
\end{lstlisting}

\end{frame}

\begin{frame}[fragile]
	\frametitle{Intestazioni}
	
\begin{lstlisting}
\documentclass[12pt,a4paper]{article}

\begin{document}
\tableofcontents
\section{Introduzione}
Testo...

\subsection{Premesse}
Testo...

\end{document}
\end{lstlisting}

\end{frame}

\begin{frame}[fragile]
	\frametitle{Titolo}
	
\begin{lstlisting}
\documentclass{article}
\title{Introduzione alla linguistica}
\author{Mario Rossi}
\date{}

\begin{document}
\maketitle
...

\end{lstlisting}

\end{frame}
\subsection{Pacchetti e formattazione}
\begin{frame}[fragile]
	\frametitle{Pacchetti}
	
\begin{lstlisting}
\documentclass{article}

\usepackage{fontspec}
  \setmainfont{Arial Unicode MS}
\usepackage{polyglossia}
  \setmainlanguage{italian}
...
\end{lstlisting}

\end{frame}

\begin{frame}[fragile]
	\frametitle{Formattazione del testo}
	
\begin{lstlisting}
Per un paragrafo nuovo, lascia uno spazio.

Questo è un nuovo paragrafo. Per formattare il testo: \textit{corsivo}, \textbf{grassetto}.

Per una nota a piè di pagina.\footnote{Questa è una nota a piè di pagina}
\end{lstlisting}

\end{frame}

\begin{frame}[fragile]
	\frametitle{Elenchi puntati e numerati}
\begin{lstlisting}
\begin{itemize}
\item Genere: maschile, femminile
\item Numero
  \begin{enumerate}
  \item singolare
  \item plurale
  \end{enumerate}
\end{itemize}
\end{lstlisting}
\end{frame}

\begin{frame}
	\frametitle{Elenchi puntati e numerati}
\begin{itemize}
\item Genere: maschile, femminile
\item Numero
  \begin{enumerate}
  \item singolare
  \item plurale
  \end{enumerate}
\end{itemize}
\end{frame}




\section{Linguistica}
\subsection{Elementi comuni}
\begin{frame}[fragile]
	\frametitle{Esempi numerati}
\begin{lstlisting}
\usepackage{gb4e}
...
\begin{exe}
\ex Nel mezzo del cammin di nostra vita
\ex
  \begin{xlist}
  \ex Mi ritrovai per una selva oscura
  \ex Che la diritta via era smarrita
  \end{xlist}
\end{exe}
\end{lstlisting}
\end{frame}

\begin{frame}
	\frametitle{Esempi numerati}
\begin{exe}
\ex Nel mezzo del cammin di nostra vita
\ex
  \begin{xlist}
  \ex Mi ritrovai per una selva oscura
  \ex Che la diritta via era smarrita
  \end{xlist}
\end{exe}
\end{frame}

\begin{frame}[fragile]
	\frametitle{Tabelle}
	
\begin{lstlisting}
\usepackage{ctable}
...
\ctable[caption=Objects,
label=obj,
pos=t
]{lcc}{}{
\FL
 & Given O & New O \ML
VOP & 85 & 65 \NN
VPO & 100 & 147 \LL
}
\end{lstlisting}

\end{frame}

\begin{frame}
	\frametitle{Tabelle}

\ctable[caption=Objects,
label=obj,
pos=t
]{lcc}{}{
\FL
 & \textsc{Given O} & \textsc{New O} \ML
VOP & 85 & 65 \NN
VPO & 100 & 147 \LL
}
\end{frame}

\begin{frame}[fragile]
	\frametitle{Glosse interlineari}
\begin{lstlisting}
\begin{exe}
\ex \gls This is a glossed example. \\
questo è un glossato esempio \\
\glt "Questo è un esempio glossato."
\end{exe}
\end{lstlisting}
\end{frame}

\begin{frame}
	\frametitle{Glosse interlineari}

\begin{exe}
\ex \gll This is a glossed example. \\
questo è un glossato esempio \\
\glt "Questo è un esempio glossato."
\end{exe}
\end{frame}


\subsection{Funzioni speciali}

\begin{frame}[fragile]
	\frametitle{Riferimenti incrociati}
\begin{lstlisting}
\section{Metodologia}
\label{meto}
Testo...
\begin{exe}
\ex\label{es} Stay hungry, stay foolish.
\end{exe}

\section{Conclusioni}
Come detto nella sezione \ref{meto}... Nell'esempio \ref{es}...

\end{lstlisting}
\end{frame}

\begin{frame}[fragile]
	\frametitle{Bibliografia}
\begin{lstlisting}
\usepackage{natbib}
\setcitestyle{aysep={},notesep={:}}
...
Come suggerito da \citet[45--47]{dixon2001intro}... Questo è stato confermato da molti studi, ai quali rimando \citep{dryer13,goldsmith1986auto,givon2002syntax}...
...
\bibliography{mybib}
\bibliographystyle{unified}
\end{lstlisting}


\end{frame}






\end{document}